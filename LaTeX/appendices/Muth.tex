\newcommand{\econtexRoot}{..}
% The \commands below are required to allow sharing of the same base code via Github between TeXLive on a local machine and Overleaf (which is a proxy for "a standard distribution of LaTeX").  This is an ugly solution to the requirement that custom LaTeX packages be accessible, and that Overleaf prohibits symbolic links
\providecommand{\econtex}{\econtexRoot/Resources/texmf-local/tex/latex/econtex}
\providecommand{\econark}{\econtexRoot/Resources/texmf-local/tex/latex/econark}
\providecommand{\econtexSetup}{\econtexRoot/Resources/texmf-local/tex/latex/econtexSetup}
\providecommand{\econtexShortcuts}{\econtexRoot/Resources/texmf-local/tex/latex/econtexShortcuts}
\providecommand{\econtexBibMake}{\econtexRoot/Resources/texmf-local/tex/latex/econtexBibMake}
\providecommand{\econtexBibStyle}{\econtexRoot/Resources/texmf-local/bibtex/bst/econtex}
\providecommand{\econtexBib}{economics}
\providecommand{\notes}{\econtexRoot/Resources/texmf-local/tex/latex/handout}
\providecommand{\handoutSetup}{\econtexRoot/Resources/texmf-local/tex/latex/handoutSetup}
\providecommand{\handoutShortcuts}{\econtexRoot/Resources/texmf-local/tex/latex/handoutShortcuts}
\providecommand{\handoutBibMake}{\econtexRoot/Resources/texmf-local/tex/latex/handoutBibMake}
\providecommand{\handoutBibStyle}{\econtexRoot/Resources/texmf-local/bibtex/bst/handout}

\providecommand{\FigDir}{\econtexRoot/Figures}
\providecommand{\CodeDir}{\econtexRoot/Code}
\providecommand{\DataDir}{\econtexRoot/Data}
\providecommand{\SlideDir}{\econtexRoot/Slides}
\providecommand{\TableDir}{\econtexRoot/Tables}
\providecommand{\ApndxDir}{\econtexRoot/Appendices}

\providecommand{\ResourcesDir}{\econtexRoot/Resources}
\providecommand{\rootFromOut}{..} % APFach back to root directory from output-directory
\providecommand{\LaTeXGenerated}{\econtexRoot/LaTeX} % Put generated files in subdirectory
\providecommand{\econtexPaths}{\econtexRoot/Resources/econtexPaths}
\providecommand{\LaTeXInputs}{\econtexRoot/Resources/LaTeXInputs}
\providecommand{\LtxDir}{LaTeX/}
\providecommand{\EqDir}{Equations} % Put generated files in subdirectory

 \documentclass{\econtex}\usepackage{\econtexSetup}\usepackage{\econtexShortcuts}% Resources that need to be accessible from various different locations

\providecommand{\YLevBF}{\ensuremath{\mathbf{Y}}}
\renewcommand{\pDies}{\ensuremath{\mathsf{d}}}
%\renewcommand{\PDies}{\ensuremath{\mathsf{D}}}
%\renewcommand{\PLives}{\ensuremath{(1-\mathsf{D})}}

\providecommand{\aggrSavingRuleCoeff}{\ensuremath{\kappa}}
\providecommand{\stnRatio}{\ensuremath{\tau}}

\newcommand{\TabsDir}{\econtexRoot/Tables}\newcommand{\Calibration}{\econtexRoot/Calibration}
\newcommand{\W}{\mathcal{W}}            % Include or exclude aggregate wage in individual problem - to include, should be \newcommand{\W}{W}

\provideboolean{Depreciation}
\setboolean{Depreciation}{true}
%\setboolean{Depreciation}{false}
\provideboolean{MicroDepr}
\setboolean{MicroDepr}{true}
%\setboolean{Depreciation}{false}
\newcommand{\ifDepr}{\ifthenelse{\boolean{Depreciation}}}
\newcommand{\MicroDepr}{\ifthenelse{\boolean{MicroDepr}}}

\newcommand{\RBet}{\mathbf{R}} % Bold  version is between-period rate; \mathcal version is within
\newcommand{\rBet}{\mathbf{r}} % Bold  version is between-period rate; \mathcal version is within
\newcommand{\RIn}{\mathcal{R}}
\newcommand{\rIn}{r}

\provideboolean{Growth}
\setboolean{Growth}{true}
\setboolean{Growth}{false}
\newcommand{\ifGG}{\ifthenelse{\boolean{Growth}}}  % Switch for whether to include productivity growth in formulae; useful for programming/debugging

\provideboolean{numSwitch}     % Switch that suppresses eqn numbers on slides but not in text
\setboolean{numSwitch}{true}   % Set to true  at beginning of text
%\setboolean{numSwitch}{false} % Set to false at beginning of slides
\providecommand{\ifnumSw}{\ifthenelse{\boolean{numSwitch}}{}{\nonumber}}

\provideboolean{Slides}
\setboolean{Slides}{false}
\newcommand{\ifslides}{\ifthenelse{\boolean{Slides}}}

\provideboolean{ZeroProb}
\setboolean{ZeroProb}{false}
\setboolean{ZeroProb}{true}
\newcommand{\ifZero}{\ifthenelse{\boolean{ZeroProb}}}

\provideboolean{DocVersion} % If true, produce an elaborate version of the document that contains all formulae used in the software
\setboolean{DocVersion}{false}
\setboolean{DocVersion}{true}
%\providecommand{\ifDoc}{\ifthenelse{\boolean{DocVersion}}}
\providecommand{\ifDoc}{\marginpar{\tiny{DocAndNodocVersions}}\emph}

% ExtraExplain toggles whether to include the explanation for why we can't assume \bar{\ell}=\ell
\provideboolean{ExtraExplain}
\setboolean{ExtraExplain}{false}
%\setboolean{ExtraExplain}{true}
\newcommand{\ifExplain}{\ifthenelse{\boolean{ExtraExplain}}}

\provideboolean{ShowStickyE}
\setboolean{ShowStickyE}{true}
\setboolean{ShowStickyE}{false}
\newcommand{\StickyE}{\ifthenelse{\boolean{ShowStickyE}}}

\newcommand{\eq}{\econtexRoot/Equations}
\newcommand{\EqnDir}{\econtexRoot/Equations}

\newcommand{\fm}{frictionless-$\mathbf{m}$ }
\newcommand{\sm}{sticky-$\mathbf{m}$ }

\newcommand{\ParmDir}{\econtexRoot/Calibration}
\newcommand{\TablesDir}{\econtexRoot/Tables}
\newcommand{\DirCampManVsStickyE}{\econtexRoot/Empirical/US/Results/LaTeX/tables}
\providecommand{\perc}[1]{\widetilde{#1}}
\provideboolean{StandAlone}\setboolean{StandAlone}{true}\setboolean{BigAndWide}{true}   \begin{document}\bibliographystyle{\econtexBibStyle} \setboolean{DocVersion}{false}  \newcommand{\texname}{Muth}\hfill{\tiny \today} \large
  \section{Formulae Derived from \cite{muthOptimal}}\label{appendix:Muth}

  \begin{verbatimwrite}{\econtexRoot/LaTeX/appendices/MuthPre-muthOptimal.texinput}
  \cite{muthOptimal}, pp.\ 303--304,  shows that the signal-extracted estimate of permanent income is
  \begin{eqnarray}
  \perc{\PLev}_{t} & = & v_{1}Y_{t}+v_{2}Y_{t-1}+v_{3}Y_{t-2}+...
  \end{eqnarray}
  for a sequence of $v$'s given by
  \begin{eqnarray}
    v_{k} & = & (1-\lambda_{1})\lambda_{1}^{k-1}
  \end{eqnarray}
  for $k = 1,2,3,...$.  So:
  \begin{eqnarray}
    \perc{\PLev}_{t} & = & (1-\lambda_{1})(\phantom{\YLev_{t+1}+\lambda_{1}^{0}}\YLev_{t}+\lambda_{1} \YLev_{t-1} + \lambda_{1}^{2} \YLev_{t-2} ...)
\\    \perc{\PLev}_{t+1} & = & (1-\lambda_{1})(\YLev_{t+1}+\lambda_{1} \YLev_{t} + \lambda_{1}^{2} \YLev_{t-1}+ \lambda_{1}^{3} \YLev_{t-2} ...)
\\     & = & (1-\lambda_{1})\phantom{(}\YLev_{t+1}+\lambda_{1} \underbrace{(1-\lambda_{1})(\YLev_{t} + \lambda_{1}^{2} \YLev_{t-1}+ \lambda_{1}^{3} \YLev_{t-2} ...)}_{\perc{\PLev}_{t}}
\\    & = & (1-\lambda_{1})\YLev_{t+1}+\lambda_{1} \perc{\PLev}_{t}
  \end{eqnarray}

This compares with \eqref{eq:PischkeP} in the main text
  \begin{eqnarray}
    \perc{\PLev}_{t+1} & = & \Pi \YLev_{t+1} + (1-\Pi) \perc{\PLev}_{t}
  \end{eqnarray}
so the relationship between our $\Pi$ and Muth's $\lambda_{1}$ is:
  \begin{eqnarray}
    \lambda_{1} & = & 1-\Pi
  \end{eqnarray}

  Defining the signal-to-noise ratio $\varphi = \sigma_{\pmb{\psi}}/\sigma_{\pmb{\theta}}$, starting with equation (3.10) in \cite{muthOptimal} we have
  \begin{eqnarray}
\notag    \lambda_{1} & = & 1 + (1/2) \varphi^{2} - \varphi \sqrt{1+\varphi^{2}/4}
\notag \\ (1-\Pi) & = & 1 + (1/2) \varphi^{2} - \varphi \sqrt{1+\varphi^{2}/4}
                        \notag\\ -\Pi & = & (1/2) \varphi^{2} - \varphi \sqrt{1+\varphi^{2}/4}
  \end{eqnarray}
  yielding equation \eqref{eq:muthOptimal} in the main text.
\end{verbatimwrite}
\input{\econtexRoot/LaTeX/appendices/MuthPre-muthOptimal.texinput}
  \begin{verbatimwrite}{\eq/muthOptimal.tex}
  \begin{eqnarray}
\Pi & = & \varphi \sqrt{1+\varphi^{2}/4} - (1/2) \varphi^{2} \label{eq:muthOptimal}
  \end{eqnarray}
\end{verbatimwrite}
  \begin{eqnarray}
\Pi & = & \varphi \sqrt{1+\varphi^{2}/4} - (1/2) \varphi^{2} \label{eq:muthOptimal}
  \end{eqnarray}


\bibliography{economics,\texname,\texname-Add}

\end{document}
