
After simulating a population of households using the procedure in Appendix~\ref{appendix:Simulation},
we have a history of micro observations $\left\{ \{\cLevBF_{t,i}, \pDies_{t,i}\}_{t=0}^T \right\}_{i=1}^I$
and a history of aggregate permanent productivity levels $\{ \PLev_t \}_{t=0}^T$.  Each household index
$i$ contains the history of many agents, as the agent at $i$ dies and is replaced at the beginning of any period
with $\pDies_{t,i}=1$.  Let $\tau_{i,n}$ be the $n$-th time $t$ index where $\pDies_{t,i}=1$; further define
$N_i = \sum_{t=0}^T \pDies_{t,i}$, the number of replacement events for household index $i$.

A single consumer's (normalized) discounted sum of lifetime utility is then:
\begin{equation*}
\vFunc_{i,n} =  \PLev_{\tau_{i,n}}^{\CRRA-1} \sum_{t=\tau_{i,n}}^{\tau_{i,n+1}-1} \beta^{t-\tau_{i,n}} \uFunc(\cLevBF_{t,i}).
\end{equation*}
Normalizing by aggregate productivity at birth $\PLev_t$ is equivalent to normalizing by the consumer's
total productivity at birth $\pLev_{t,i}$ because $p_{t,i}=1$ at birth by assumption.

The total number of households who are born and die in the history is:
\begin{equation*}
N_I = \sum_{i=1}^I (N_i - 1).
\end{equation*}
The overall expected lifetime value at birth can then be computed as:
\begin{equation*}
\overline{\vFunc}_0 = N_I^{-1} \sum_{i=1}^I \sum_{n=1}^{N_i-1} \vFunc_{i,n}.
\end{equation*}

Because we use $T=20,000$ and $I=20,000$, and agents live for 200 periods on average ($\PDies = 0.005$),
our simulated history includes about $N_I \approx I T \PDies =$ 2 million consumer lifetimes.  The standard errors
on our numerically calculated $\overline{\vFunc}_0$ and $\overline{\widetilde{\vFunc}}_0$ are thus negligible
and not reported.

In the SOE model, we use the same random seed for the frictionless and sticky specifications, so
the same sequence of replacement events and income shocks occurs in both. With no externalities
or general equilibrium effects, the distribution of states that consumers are born into is likewise identical,
so the ``value ratio'' calculation is valid.

The cost of stickiness in the HA-DSGE model is slightly more complicated.  If we
used the generated histories of the frictionless and sticky specifications to compute
$\overline{\vFunc}_0$ and $\overline{\widetilde{\vFunc}}_0$, the calculated $\omega$
would represent a newborn's willingness-to-pay for \textit{everyone} to be frictionless
rather than sticky. We are interested in the utility cost of \textit{just one agent} having
sticky expectations, so an alternate procedure is required.

We compute $\overline{\widetilde{\vFunc}}_0$ in the HA-DSGE model the same as in the SOE
model.  However, $\overline{\vFunc}_0$ is calculated as the expected lifetime (normalized) value
of a newborn who is frictionless \textit{but lives in a world otherwise populated by sticky consumers}. To do
this, we simulate a new history of micro observations using the consumption function for the
sticky HA-DSGE economy, but with all $I$ households updating their knowledge of the aggregate
state frictionlessly.  Critically, we \textit{do not} actually calculate $\ALevBF_t = \KLevBF_{t+1}$
each period; instead, we use the \textit{same sequence} of $\ALevBF_t$ that occurred in the
ordinary sticky simulation.  Thus our simulated population of $I$ households represents an infinitesimally
small portion of an economy made up (almost) entirely of consumers with sticky expectations.  The
calculated $\omega$ is thus the willingness-to-pay to be the very first agent to ``wake up.''

The formula for willingness-to-pay \eqref{eq:WTP} arises from the homotheticity of the household's
problem with respect to $\pLev_{t,i}$.  If a consumer gives up an $\omega$ portion of their permanent
income at the moment they are ``born'', before receiving income that period, then his normalized market
resources will still be $\mLev_{t,i} = \Wage_t$, and he will make the same normalized consumption
choice that he would have, had he not lost any permanent income.  In fact, he will make the \textit{exact same}
sequence of normalized consumption choices for his entire life; the \textit{level} of his consumption will
be scaled by the factor $(1-\omega)$ in every period.  With CRRA utility, this means that utility is scaled
by $(1-\omega)^{1-\CRRA}$ in every period of life, which can be factored out of the lifetime summation.
The indifference condition between being frictionless and losing an $\omega$ fraction of permanent income
versus having sticky expectations (and not losing) can be easily rearranged into \eqref{eq:WTP}.
