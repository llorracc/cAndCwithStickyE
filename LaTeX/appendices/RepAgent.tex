
This appendix presents a representative agent model for analyzing the consequences of sticky expectations in a DSGE framework while abstracting from idiosyncratic income shocks and the death (and replacement) of households.  It builds upon the modeling assumptions in section~\ref{sec:models} to formulate the representative agent model, then presents simulated results analogous to section~\ref{sec:Results}.  The primary advantage of this model is that it allows fast analysis of sticky expectations in a closed economy, yielding very similar results to the heterogeneous agents DSGE model with less than a minute of computation, rather than a few hours.  However, the model is not truly a ``representative agent'' model under sticky expectations; instead it is as though there is an agent whose beliefs about the aggregate state are ``smeared'' over the state space with a probability distribution that reflects the distribution of perceptual delay implied by the Calvo updating probability. That is, the ealized level of consumption represents the weighted average level of consumption chosen by the ``many minds'' of the representative household, with weights reflecting the likelihood of each possible degree of perceptual delay.


\subsection{Model and Solution}

The representative agent's state variables at the time of its consumption decision are the
level of market resources $\MLevBF_t$, the productivity of labor $\PLev_t$, and the growth
rate of productivity $\PtyGro_t$.  Idiosyncratic productivity shocks $\psi$ and $\theta$ do
not exist, and the possibility of death is irrelevant; aggregate permanent and transitory
productivity shocks $\Psi$ and $\Theta$ are distributed as usual.

The representative agent's problem can be written in Bellman form as:\footnote{Subject to definitions \eqref{eq:AggRandWalk}, \eqref{eq:Ktp1}, \eqref{eq:Lt}, \eqref{eq:Mtp1} and \eqref{eq:MargProd}.}
\begin{eqnarray*}
\textbf{V}(\MLevBF_t,\PLev_t,\PtyGro_t) &=& \max_{\CLevBF_t} \big\{ \uFunc(\CLevBF_t) + \Ex \big[\textbf{V}(\MLevBF_{t_+1},\PLev_{t+1},\PtyGro_{t+1})\big]   \big\} \nonumber
\\  & \mbox{s.t.} & \nonumber
\\ \ALevBF_t &=& \MLevBF_t - \CLevBF_t. \nonumber
\end{eqnarray*}

Normalizing the representative agent's problem by the productivity level $\PLev_t$ as in
the SOE and HA-DSGE models, the problem's state space can be reduced to:\footnote{Subject to definitions \eqref{eq:AggRandWalk}, \eqref{eq:Mtp1norm} and \eqref{eq:MargProd}.}
\begin{verbatimwrite}{\eq/RAscaledV.tex}
\begin{eqnarray}
\ifnumSw  \Value(\MLev_{t},\PtyGro_t) & = & \max_{\CLev_{t}} \big\{ \uFunc(\CLev_{t}) + \beta \Ex_{t} \big[ (\PtyGro_{t+1}\Psi_{t+1})^{1-\CRRA} \Value(\MLev_{t+1},\PtyGro_{t+1}) \big] \big\} \label{eq:RAscaledV} \ifnumSw
\\  & \mbox{s.t.} & \nonumber
\\ \ifnumSw \ALev_{t} & = & \MLev_{t}-\CLev_{t}. \nonumber
\end{eqnarray}
\end{verbatimwrite}
\input \eq/RAscaledV.tex
Noting that the return to (normalized) end-of-period assets for next period's
market resources is
$\frac{\text{d}\MLev_{t+1}}{\text{d} \ALev_{t}} = \Rprod_{t+1}/(\PtyGro_{t+1}\Psi_{t+1})$,
\eqref{eq:RAscaledV} has a single first-order condition that is sufficient
to characterize the solution to the normalized problem:
\begin{equation}\label{eq:FOCRA}
\CLev_t^{-\CRRA} - \underbrace{\beta \Ex \big[\Rprod_{t+1} (\PtyGro_{t+1}\Psi_{t+1})^{-\CRRA} \Value^\MLev(\ALev_{t} \Rprod_{t+1}/(\Psi_{t+1} \PtyGro_{t+1}) + \Theta_{t+1} \Wage_{t+1}, \PtyGro_{t+1})  \big]}_{\equiv\, \mathfrak{V}^\ALev(\ALev_t,\PtyGro_t)} = 0
\end{equation}
\begin{equation*}
\Longrightarrow \CLev_t = \mathfrak{V}^\ALev(\ALev_t,\PtyGro_t)^{-1/\CRRA}.
\end{equation*}
The representative agent model can be solved using the endogenous grid method,
following the same procedure as for the SOE model described in Appendix~\ref{appendix:Solution}, yielding
 normalized consumption function $\text{C}(\MLev,\PtyGro)$.\footnote{The only
differences in solution method are that the RA model uses $N_\Psi=N_\Theta=7$ point
approximations to the aggregate shock distribution, expected marginal value of assets
is calculated using \eqref{eq:FOCRA}, and the upper bound of $\mathbb{A}$ is 120.}


\subsection{Frictionless vs Sticky Expectations}

The typical interpretation of a representative agent model is that
it represents a continuum of households that face no idiosyncratic shocks, and
thus all find themselves with the same state variables; idiosyncratic decisions
are equivalent to aggregate, representative agent decisions.  Once we introduce
sticky expectations of aggregate productivity, this no longer holds: different
households will have different perceptions of productivity, and thus make
different consumption decisions.

To handle this departure from the usual representative agent framework,
we take a ``multiple minds'' or quasi-representative agent approach.  That is,
we model the representative agent as being made up of a continuum of
households who all correctly perceive the level of aggregate market
resources $\MLevBF_t$, but have different perceptions of the aggregate productivity
state.  Each household chooses their level of consumption based on their
perception of the productivity state; the realized level of aggregate consumption
is simply the sum across all households.

Formally, we track the \textit{distribution} of perceptions about the aggregate productivity
state as a stochastic vector $\varphi_t$ over the current growth rate $\PtyGro_t \in \{\PtyGro\}$,
representing the fraction of households who perceive each value of $\PtyGro$,
and a vector $\perc{\PLev}_t$ representing the average perceived productivity
level among households who perceive each $\PtyGro$.  As in our other models,
agents update their perception of the true aggregate
productivity state $(\PLev_t,\PtyGro_t)$ with probability $\Pi$; likewise, the
distinction between frictionless and sticky expectations is simply whether $\Pi=1$
or $\Pi < 1$.

Defining $e^j_N$ as the $N$-length vector with zeros in all elements but the
$j$-th, which has a one, the distribution of population perceptions of growth
rate $\PtyGro_t$ evolves according to:
\begin{equation}\label{eq:PtyGroPerc}
\varphi_{t+1} = (1 - \Pi) \varphi_{t} + \Pi e_N^j \text{ when } \PtyGro_{t+1} = \PtyGro_j.
\end{equation}
That is, a $\Pi$ proportion of households who perceive each growth rate update
their perception to the true state $\PtyGro_{t+1} = \PtyGro_j$, while the other $(1-\Pi)$
proportion of households maintain their prior belief (which might already be $\PtyGro_j$).

The vector of average perceptions of aggregate productivity for each growth
rate can then be calculated as:
\begin{equation}\label{eq:PLevPercRA}
\perc{\PLev}_{t+1} = \big((1 - \Pi) \varphi_{t} \odot \perc{\PLev}_t + \Pi e_N^j \PLev_{t+1} \big) \oslash \varphi_{t+1}.
\end{equation}
That is, the average perception of productivity in each growth state is the weighted
average of updaters and non-updaters who perceive that growth rate.\footnote{The
Hadamard operators $\odot$ and $\oslash$ represent element-wise multiplication and
division, respectively. As a numeric detail, $\perc{\PLev}_t^j$ is reset to 1 when $\varphi_t^j=0$,
which would otherwise cause it to be undefined.  When no households perceive growth
rate $\PtyGro_j$, the average perception of productivity does not exist for this state and is quantitatively
irrelevant, but must exist for \eqref{eq:PLevPercRA} to not fail in the next period.}

Households who perceive each growth rate $\PtyGro$ choose their level of consumption according to their perception of normalized market resources, as though they knew their perception to be the truth.  Defining $\perc{\MLev}_t^j = \MLevBF_t/\perc{\PLev}_t^j$ as perceived normalized market resources for households who perceive the aggregate growth rate is $\PtyGro_j$, aggregate consumption is:
\begin{equation}\label{eq:cLvlRA}
\CLevBF_t = \sum_{\PtyGro_j \in \{\PtyGro\}} \perc{\PLev}_t^j \text{C}(\perc{\MLev}_t^j,\PtyGro_j) \varphi_t^j.
\end{equation}
This represents the weighted average of per-state consumption levels of the partial
representative agents.

When the representative agent frictionlessly updates its information every period ($\Pi=1$),
equations \eqref{eq:PtyGroPerc} and \eqref{eq:PLevPercRA} say that $\varphi_t = e_N^j$
and $\perc{\PLev}_t^j = \PLev_t$ (with irrelevant values in the other vector elements), so
that the representative agent is truly representative.  When expectations are sticky ($\Pi < 1$),
the representative agent's perceptions of the growth rate become ``smeared'' across its
past realizations; its perceptions the productivity level likewise deviate from the true value,
even for the part of the representative agent who perceives the true growth rate.\footnote{An alternative method for modeling sticky expectations with a representative agent would be to track the
perceptions of the segments of households who last updated $n=\{0,1,\ldots,200\}$
periods ago, compute consumption for each segment, and take the weighted average
across the segments to yield aggregate consumption.  This approach would only be
slightly more complicated to implement, and we believe it would yield quantitatively
similar results.}


\subsection{Simulation Results}

\input\econtexRoot/Tables/RepAgentMrkvSimReg.tex

We calibrate the RA model using the same parameters as for the
HA-DSGE model (see Appendix~\ref{sec:MacroCal}, Table~\ref{table:calibration}, and Appendix~\ref{appendix:HADSGEresults}),
except that there are no idiosyncratic income shocks
($\sigma^2_\psi = \sigma^2_\theta = \wp = 0$) and the possibility
of death is irrelevant ($\PDies=0$).  After solving the model, we
utilize the same simulation procedure described in section~\ref{sec:Results},
taking 100 samples of 200 quarters each; average coefficients
and standard errors across the samples are reported in Table~\ref{tRAsim}.

The upper panel of Table~\ref{tRAsim} shows that under frictionless
expectations, consumption growth in the representative agent model
cannot be predicted to any statistically significant degree under any
specification.  The lower panel, under sticky expectations, yields
results that are strikingly similar to the SOE model in Table~\ref{tPESOEsim}.
Both (instrumented) lagged consumption growth and expected income
growth are significant predictors of aggregate consumption growth,
but the `horse race' regression reveals that the predictability is
dominated by serially correlated consumption growth, confirming the
results of the two heterogeneous agents models.
