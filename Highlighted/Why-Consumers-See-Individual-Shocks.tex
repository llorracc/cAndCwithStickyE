
  One reason we assume that both frictionless and sticky-expectations consumers can perceive the idiosyncratic components of their income (the $\pRat$ and $\theta$) is that this is the assumption made by almost all of the `modern' literature, and therefore makes our paper's results easily comparable with that literature.

  But the assumption can be defended on its own terms; it is consistent with evidence from a number of sources.

  First, there are at least some shocks whose transitory nature is impossible to misperceive; the best example is lottery winnings in Norway, see again \cite{fhnMPC}.  The consumption responses to those shocks resemble the responses measured in the previous literature to shocks that economists presumed that consumers knew to be transitory.  If consumers respond to such shocks in ways similar to their responses to unambiguously transitory shocks like lottery winnings, that would seem to support the proposition that consumers correctly perceive as transitory those other shocks that economists have presumed consumers identified as transitory.

  Second, one reason to believe that perception of the idiosyncratic permanent shocks is not difficult comes from \cite{lmpPermShocks}, who show that a large proportion of permanent shocks to income occur at the times of job transitions (mostly movements from one job to another).  It would be hard to believe that consumers switching jobs were not acutely aware of the difference between the incomes yielded by those two jobs.

  Earlier work by \cite{pistaferriSuperior} developed a method for decomposing income shocks into permanent and transitory components.  He finds that data from a survey in which consumers are explicitly asked about their income expectations provides a powerful tool to estimate the magnitude of permanent versus transitory shocks; relatedly, \cite{gsInferring} find that consumption choices provide important information about subsequent income movements.

  More direct and more recent evidence comes from \cite{kmpIncomeExpectations}.  Using data from the New York Fed's Survey of Consumer Expectations (SCE), they find that on average, the difference between four-month-ahead realizations of household income and four-month-ahead expectations is near zero and the average error is only 0.5 percent. \cite{kmpIncomeExpectations} explicitly interpret their evidence from the survey as suggesting that consumers have accurate perceptions of the permanent and transitory components of their income.

  A final bit of evidence comes from metadata associated with the \textit{Survey of Consumer Finances}, which asks a question designed to elicit consumers' perceptions of their permanent (``usual'') income.  A well-known fact in among survey methodologists is that the speed and ease with which consumers answer a question is an indicator of the extent to which they have a clear understanding of the question and are confident in their answer.  The SCF question designed to elicit consumers perceptions of their permanent income is an example of such a question: Consumers answer quickly and easily and do not seem to exhibit any confusion about what they are being asked (\cite{kennickellPermanent}).

  In contrast, we are aware of no corresponding evidence that consumers are well informed about aggregate income (especially at high frequencies).%
